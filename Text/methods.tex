% Options for packages loaded elsewhere
\PassOptionsToPackage{unicode}{hyperref}
\PassOptionsToPackage{hyphens}{url}
%
\documentclass[
]{article}
\usepackage{lmodern}
\usepackage{amssymb,amsmath}
\usepackage{ifxetex,ifluatex}
\ifnum 0\ifxetex 1\fi\ifluatex 1\fi=0 % if pdftex
  \usepackage[T1]{fontenc}
  \usepackage[utf8]{inputenc}
  \usepackage{textcomp} % provide euro and other symbols
\else % if luatex or xetex
  \usepackage{unicode-math}
  \defaultfontfeatures{Scale=MatchLowercase}
  \defaultfontfeatures[\rmfamily]{Ligatures=TeX,Scale=1}
\fi
% Use upquote if available, for straight quotes in verbatim environments
\IfFileExists{upquote.sty}{\usepackage{upquote}}{}
\IfFileExists{microtype.sty}{% use microtype if available
  \usepackage[]{microtype}
  \UseMicrotypeSet[protrusion]{basicmath} % disable protrusion for tt fonts
}{}
\makeatletter
\@ifundefined{KOMAClassName}{% if non-KOMA class
  \IfFileExists{parskip.sty}{%
    \usepackage{parskip}
  }{% else
    \setlength{\parindent}{0pt}
    \setlength{\parskip}{6pt plus 2pt minus 1pt}}
}{% if KOMA class
  \KOMAoptions{parskip=half}}
\makeatother
\usepackage{xcolor}
\IfFileExists{xurl.sty}{\usepackage{xurl}}{} % add URL line breaks if available
\IfFileExists{bookmark.sty}{\usepackage{bookmark}}{\usepackage{hyperref}}
\hypersetup{
  pdftitle={Methods},
  pdfauthor={Emilio Akira Morones Ishikawa},
  hidelinks,
  pdfcreator={LaTeX via pandoc}}
\urlstyle{same} % disable monospaced font for URLs
\usepackage[margin=1in]{geometry}
\usepackage{graphicx}
\makeatletter
\def\maxwidth{\ifdim\Gin@nat@width>\linewidth\linewidth\else\Gin@nat@width\fi}
\def\maxheight{\ifdim\Gin@nat@height>\textheight\textheight\else\Gin@nat@height\fi}
\makeatother
% Scale images if necessary, so that they will not overflow the page
% margins by default, and it is still possible to overwrite the defaults
% using explicit options in \includegraphics[width, height, ...]{}
\setkeys{Gin}{width=\maxwidth,height=\maxheight,keepaspectratio}
% Set default figure placement to htbp
\makeatletter
\def\fps@figure{htbp}
\makeatother
\setlength{\emergencystretch}{3em} % prevent overfull lines
\providecommand{\tightlist}{%
  \setlength{\itemsep}{0pt}\setlength{\parskip}{0pt}}
\setcounter{secnumdepth}{-\maxdimen} % remove section numbering
\usepackage{physics}
\usepackage{float}
\usepackage[]{natbib}
\bibliographystyle{plainnat}

\title{Methods}
\author{Emilio Akira Morones Ishikawa}
\date{7/22/2020}

\begin{document}
\maketitle

Survival analysis is an inferential technique that aims to model the
time it takes for an event to occur between two events called
\emph{start time} and \emph{end time}. Although it is not related to any
particular event (which is why its application is multiple), the purpose
of this dissertation is to treat it within the clinical context: that
is, we will analyze the time necessary, depending on the disease, a
patient in some cases relapse. There is a possibility that not all
patients will relapse, so we will say that the patient cured.

These methods are mostly rooted in two main formulations: standard (mix)
model with cure rate from \citet{Berkson1952} and an alternative model
(proportional risk) with cure rate from \citet{Hoang1996a}. It was then
studied by \citet{Chen1999a} in the Bayesian context. The standard model
with cure rate \citep{Berkson1952} is a mix of survival functions
defined by:

\begin{align} \label{mixS}
S_{pop}(t)=\pi+(1-\pi)S(t)
\end{align}

where \(\pi \in (0,1)\) and \(S (t)\) have the characteristics of a
survival function. \(S_ {pop} (t)\) is a mix between a survival function
that is always worth 1 and another that decays to zero, \(S (t)\)and the
probability of belonging to one and the other is \(\pi\) and
\(1 - \pi\). Note that \(\lim \limits_{t \to 0} S_{pop} (t) = 1\) and
\(\lim \limits_{t \to \infty} S_ {pop} (t) = \pi> 0\), therefore
\(S_ {pop} (t)\) is an improper survival function. If we ignore the
impropriety of \(S_ {pop} (t)\) and continue with the traditional
relationship between a survival function and the cumulative risk
function, we have

\[
H_{pop}(t)=-\log{S_{pop}(t)}=-\log\{\pi+(1-\pi)S(t)\}
\]

which satisfies that \(\lim \limits_{t \to0} H_{pop} (t) = 0\) and
\(\lim \limits_{t \to \infty} H_{pop} (t) = - \log {\pi }> 0\). If we
proceed and calculate the first derivative, then the risk rate is

\[
h_{pop}(t) = \dv{}{t}H_{pop}(t)=\frac{(1-\pi)f(t)}{\pi+(1-\pi)S(t)}
\]

where \(f (t)\) is the density function corresponding to \(S (t)\). The
properties of \(h_{pop} (t)\) are as follows:
\(\lim \limits_{t \to 0} h_{pop} (t) = (1- \pi) f (0)\) and
\(\lim \limits_{t \to \infty} h_{pop} (t) = 0\).

The alternative model with the cure rate of \citet{Hoang1996a} is
defined by

\begin{align} \label{supalter}
S_{pop}(t)=\exp\{-\theta F(t)\}
\end{align}

where \(\theta > 0\) and \(F (t)\) is a cumulative distribution. This
model satisfies the conditions
\(\lim \limits_{t \to 0} S_{pop} (t) = 1\) and
\(\lim \limits_{t \to \infty} S_{pop} (t) = e^{-\theta}\), therefore
\(S_ {pop} (t)\) is not a survival function per se. What it means is
that the density function associated with the cure model integrates
\(e^{- \theta}\) from the cure time to infinity, that is,
\(\int_s ^ \infty f (t) dt = e^{- \theta}\).

Figures \ref{fig:compsuper} and \ref{fig:comparativo} show the survival
and risk functions, respectively, with and without the cure threshold of
\citet{Hoang1996a} for the Weibull distribution. So
\(S(x) = \exp (-\lambda x \alpha)\) and
\(h (t) = \alpha \lambda t^{\alpha - 1}\) with \(\lambda > 0\),
\(\alpha > 0\).

\begin{figure}[H]
\includegraphics[width=1\linewidth]{methods_files/figure-latex/compsuper-1} \caption{Comparison of Weibull survival functions with cure threshold}\label{fig:compsuper}
\end{figure}

In Figure \ref{fig:compsuper}, the solid black line represents the
Weibull survival function for \(\lambda = 0.00208\) and \(\alpha = 3\),
the solid red line represents the Weibull survival function with rate
\citet{Hoang1996a} cure with the same parameters. The dotted black line
represents \(\lim_{t \to \infty} S_{pop} (t) = e^{-.9}\). What it
implies is that the individual will not present the failure.

Despite being improper, the accumulated risk function is

\[
H_{pop}(t)=-\log S_{pop}(t)=\theta F(t)
\]

which takes us to \(\lim_{t \to 0} H_{pop} (t) = 0\) and
\(\lim_ {t \to \infty} H_ {pop} (t) = \theta> 0\), if we take the first
derivative then the risk rate is

\[
h_{pop}(t)=\dv{}{t}H_{pop}(t)=\theta f(t)
\]

where \(f (t)\) is the density function that corresponds to \(F (t)\).
We have that \(\lim \limits_{t \to 0} h_ {pop} (t) = \theta f (0)\) and
\(\lim \limits_{t \to \infty} h_{pop} (t) = 0\).

\begin{figure}[H]
\includegraphics[width=1\linewidth]{methods_files/figure-latex/comparativo-1} \caption{Comparison of Weibull risk function with cure threshold}\label{fig:comparativo}
\end{figure}

In Figure \ref{fig: comparativo}, the solid black line represents the
Weibull risk function for \(\lambda = 0.00208\) and \(\alpha = 3\). The
solid red line represents the Weibull risk function with the cure rate
of \citet{Hoang1996a} with the same parameters. It implies that from a
point in time, the risk rate decreases until it reaches zero.

The semi-parametric model of the cure rate usually focuses on modeling
the risk rate function \(h (t)\) of the group that is not cured,
corresponding to \(S (t)\) in the equation \eqref{mixS} and \(F(t)\) in
\eqref{supalter}. \citet{Tsodikov2003a} proposed a piecewise constant
risk function in the model for uncured individuals,

\begin{align} \label{risk}
h(t)=\sum_{j=1}^J\lambda_j I(s_{j-1}< t \leq s_j)
\end{align}

where \(0 <s_1 <\dots <s_J\) with \(s_J> t_i\) for \(i = 1, \dots, n\).
The number \(J\) controls the degree of flexibility of the model. If
\(J = 1\) , \(h (t)\) is a constant risk rate and the larger \(J\) is,
the more flexible the model becomes.

In all models of cure in the literature, the researchers model a
positive probability of cure. However, they do not explicitly quantify
the finite cure time. \citet{Nieto-Barajas2008} proposed a new cure rate
model that allows a separation of the cured group of the uncured by
explicitly modelling the finite cure threshold. For this purpose, they
define a risk rate with a cure threshold for the entire population,
including cured and uncured individuals. A mixed gamma process
establishes the initial nonparametric distribution for the risk rates. A
mixture between a gamma distribution and a point of mass zero defines
the marginal distribution in each partition. In particular, the new
model allows determining the specific threshold to considerate an
individual cured.

\hypertarget{model-with-cure-threshold}{%
\subsection{\texorpdfstring{Model with cure threshold
\label{cap:mod_cura}}{Model with cure threshold }}\label{model-with-cure-threshold}}

The conditions that require a risk function of the entire population, in
the cure models mentioned above, are that they satisfy the following:

\begin{itemize} 
\item $\lim\limits_{t\to0}H_{pop}(t)=0$
\item $\lim\limits_{t\to\infty}H_{pop}(t)= c < \infty$
\end{itemize}

If we define \(h_{pop} = \dv{} {t} H_{pop} (t)\), for these conditions
to be fulfilled it is necessary that
\(\lim \limits_{t \to \infty} h_{pop} (t) = 0\).

After enough time, uncured individuals will experience the event, while
the cured will either be censored or remain alive in the study. That is,
after the cure threshold of the remaining individuals in the study is no
longer at risk of experiencing the event. Following this route,
\citet{Nieto-Barajas2008} propose a new cure model for the risk function
of the entire population that disappears when \(t\) exceeds a certain
threshold, \(\tau\):

\begin{align} \label{modelnieto}
h_{pop}(t)=h(t)I(t\leq\tau)
\end{align}

with \(h(t)\) a non-negative function. This new specification of the
risk function can be interpreted as a mixed cure model as in
\eqref{mixS} or a proportional cure model \eqref{supalter} limiting the
event time of the uncured group, \(T^u\) , on the right with the
threshold \(\tau\), \(Pr (T^u \leq \tau) = 1\). In practice, it is more
realistic to determine that the risk rate falls to zero after the cure
threshold since the risk of individuals who have survived to time
\(\tau\) becomes nil. The \eqref{modelnieto} model defines a cure model
that allows identifying the two groups, cured and uncured completely. In
the Bayesian context, \citet{Nieto-Barajas2008} define a nonparametric
initial distribution.

\[
h(t)=\sum_{k=1}^\infty \lambda_k I(\tau_{k-1}<t\leq \tau_k)
\]

where \(0 = \tau_0 <\tau_1 <\dots\) form a partition on the time axis
and \(\{\lambda_k\}\) is an independent range process for time,
i.e.~\(\lambda_k {\sim} Ga (\alpha_k, \beta_k)\) denotes a Gamma
distribution with mean \(\frac{\alpha_k}{\beta_k}\). Although this model
is defined with an infinite number of intervals, in practice it is
limited to a finite number. If we combine this process and the
constrained model \eqref{modelnieto}, the cure time \(\tau\) can only
occur in discrete periods, which can be estimated with a fine preference
partition to obtain greater flexibility in the model. Denoting
\(\tau_z\) as the discretized cure time, then the condition
\(t \leq \tau_z\) can be replaced by \(k \leq z\), and then the initial
distribution of the entire population is

\begin{align}\label{apriori}
h_{pop}(t)=\sum_{k=1}^\infty \lambda_k I(k\leq z)I(\tau_{k-1}<t\leq\tau_k)
\end{align}

Furthermore, we can take an initial distribution for \(\tau_z\)
considering one for \(z\). If we denote the initial distribution for
\(z\) as \(f(z)\), then the new process \(\{\lambda_k^*\}\) with
\(\lambda_k^* = \lambda_k I (k \leq z)\), is characterized by

\[
f(\lambda_k^*\mid z) = Ga(\lambda_k^* \mid \alpha_k,\beta_k)I(k\leq z)+I(\lambda_k^*=0)I(k>z)
\]

Marginalizing over \(z\), the initial distribution of \(\lambda_k^*\)
becomes

\[
f(\lambda_k^*) = \eta_k Ga(\lambda_k^* \mid \alpha_k,\beta_k) + (1-\eta_k)I(\lambda_k^*=0)
\]

with \(\eta_k = Pr(z \geq k)\), i.e.~\(\lambda_k^*\) has a prior
distribution given by mixing a gamma distribution and a point of mass at
zero \(1- \eta_k\).

The first and second moment of the process, \(\{\lambda_k^*\}\) are the
following:

\[
E(\lambda_k^*)=\eta_k\bigg(\frac{\alpha_k}{\beta_k}\bigg) \quad
Var(\lambda_k^*)=\eta_k(1-\eta_k)\bigg(\frac{\alpha_k}{\beta_k}\bigg)^2+\eta_k\bigg(\frac{\alpha_k}{\beta_k^2}\bigg).
\]

Due to the existence of \(z\), \(\lambda_k^*\) and \(\lambda_{k + 1}^*\)
are not independent, and the covariance is given by

\[
Cov(\lambda_{k+1}^*,\lambda_k^*) = \eta_{k+1}(1-\eta_k)\bigg(\frac{\alpha_{k+1}}{\beta_{k+1}}\bigg)\bigg(\frac{\alpha_{k}}{\beta_{k}}\bigg).
\]

It is important to note that the equation \eqref{modelnieto} allows
modelling non-parametrically the risk rates for the entire population,
while other model constructs for the cure rate model the cure rate for
uncured individuals. If \(\eta_k = 1\), the time in interval \(k\)
occurs before cure index \(z\),

\[
E(\lambda_k^*)=\frac{\alpha_k}{\beta_k} \quad
Var(\lambda_k^*)=\frac{\alpha_k}{\beta_k^2} \quad
Cov(\lambda_{k+1}^*,\lambda_k^*) =0
\]

Therefore it is reduced to the independent exponential model by
intervals.

It is convenient to add a dependency process between the risk rates of
consecutive intervals to reduce the possibility of abrupt changes in the
estimate.

To consider greater dependency on the \$\{\lambda\_k\} \$ process, we
take the Markov gamma process proposed by @ Nieto-Barajas2002a for the
\(\lambda_k\) 's. The Markov gamma process definition is through a
latent process \(\{u_k\}\) as follows. The steps are the following.

\begin{enumerate}
\item $\lambda_1 \sim Ga(\alpha_1,\beta_1)$
\item $u_k \mid \lambda_k \sim Po(c_k \lambda_k)$
\item $\lambda_{k+1}|u_k \sim Ga(\alpha_{k+1}+u_k,\beta_{k+1} + c_k)$
\end{enumerate}

for \(k = 1,2, \dots\). Here, \(Po(c)\) denotes a Poisson distribution
with intensity \(c\). If \(\alpha_k\) and \(\beta_k\) are taken as
constants, then the process \$\{~lambda\_k \} \$ is stationary with
marginal \(\lambda_k \sim Ga(\alpha, \beta)\), with correlation
\(Corr(\lambda_{k + 1}, \lambda_k) = \frac{c_k}{\beta + c_k}\).

If \(z \ geq k\) \(\forall k\) with probability 1, there is no finite
cure time, the initial \eqref{apriori} is reduced to the case
\citet{Nieto-Barajas2002a}, If \(c_k = 0\) \(\forall k\), then
\(u_k = 0\) with probability 1, this implies that \(\{\lambda_k^*\}\) is
reduced to the case where the dependency is low.

The equation \eqref{apriori} implies that the cure fraction \(\pi\),
defined as the proportion of the population that will never experience
failure, is given by

\[
\pi=\lim\limits_{t\to\infty}S_{pop}(t)=\exp\bigg\{-\sum_{k=1}^z \lambda_k(\tau_k-\tau_{k-1})\bigg\},
\]

Since \(\lambda_k = 0\) when \(k> z\) and, by definition, the risk is
null when the individual survives the healing time, the sum in the
expression \(\pi\) is a finite sum bounded by \(z\).

A pre-distribution is specified for \(z\), supported at
\(\{1,2, \dots \}\). The natural option is to consider a positive
Poisson distribution, \(z \sim Po^+ (\mu)\), for \(\mu> 0\),
i.e.~\(x-1 \sim Po (\mu)\). Although all the paths of the previous risk
function, \(h_{pop}(t)\), are piecemeal constant, the paths of the
corresponding cumulative risk function, \(H_{pop} (t)\), are continue
with probability 1. So, the nonparametric previous,
\(S_{pop} (t) = e^{- H_{pop} (t)}\), assigns a positive probability for
the set of continuous survival functions, this allows a finite healing
time.

\hypertarget{posterior-analisis}{%
\section{Posterior analisis}\label{posterior-analisis}}

\hypertarget{description-of-the-method}{%
\section{Description of the method}\label{description-of-the-method}}

\hypertarget{description-of-the-data}{%
\section{Description of the data}\label{description-of-the-data}}

\hypertarget{results}{%
\section{Results}\label{results}}

\renewcommand\refname{Discussion}
  \bibliography{diss.bib,extra.bib}

\end{document}
